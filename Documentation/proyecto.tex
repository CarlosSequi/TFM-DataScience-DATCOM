\documentclass[a4paper,11pt]{book}
%\documentclass[a4paper,twoside,11pt,titlepage]{book}
\usepackage{listings}
\usepackage[utf8]{inputenc}
\usepackage[spanish]{babel}
\usepackage{float}
\usepackage{amsmath}
\usepackage{algorithm}
\usepackage[noend]{algpseudocode}
\makeatletter
\def\BState{\State\hskip-\ALG@thistlm}
\makeatother

% \usepackage[style=list, number=none]{glossary} %
%\usepackage{titlesec}
%\usepackage{pailatino}

\decimalpoint
\usepackage{dcolumn}
\newcolumntype{.}{D{.}{\esperiod}{-1}}
\makeatletter
\addto\shorthandsspanish{\let\esperiod\es@period@code}
\makeatother


%\usepackage[chapter]{algorithm}
\RequirePackage{verbatim}
%\RequirePackage[Glenn]{fncychap}
\usepackage{fancyhdr}
\usepackage{graphicx}
\usepackage{afterpage}

\usepackage{longtable}

\usepackage[pdfborder={000}]{hyperref} %referencia

% ********************************************************************
% Re-usable information
% ********************************************************************
\newcommand{\myTitle}{Árboles de clasificación monotónica sobre flujos de datos.}
\newcommand{\myTitleEng}{Monotonic classification trees on data streams.}
\newcommand{\myDegree}{Máster DATCOM: Ciencia de Datos}
\newcommand{\myName}{Carlos Manuel Sequí Sánchez}
\newcommand{\myProf}{Salvador García López}
\newcommand{\myFaculty}{Escuela Técnica Superior de Ingenierías Informática y de
Telecomunicación}
\newcommand{\myFacultyShort}{E.T.S. de Ingenierías Informática y de
Telecomunicación}
\newcommand{\myDepartment}{Departamento de Ciencias de la Computación e I.A.}
\newcommand{\myUni}{\protect{Universidad de Granada}}
\newcommand{\myLocation}{Granada}
\newcommand{\myTime}{\today}
\newcommand{\myVersion}{Version 0.1}

\hypersetup{
pdfauthor = {\myName (email (en) ugr (punto) es)},
pdftitle = {\myTitle},
pdfsubject = {},
pdfkeywords = {Monotonic, data streams, decision tree, Hoeffding tree, classification, MOA},
pdfcreator = {LaTeX con el paquete ....},
pdfproducer = {pdflatex}
}

%\hyphenation{}


%\usepackage{doxygen/doxygen}
%\usepackage{pdfpages}
\usepackage{url}
\usepackage{colortbl,longtable}
\usepackage[stable]{footmisc}
%\usepackage{index}

%\makeindex
%\usepackage[style=long, cols=2,border=plain,toc=true,number=none]{glossary}
% \makeglossary

% Definición de comandos que me son tiles:
%\renewcommand{\indexname}{Índice alfabético}
%\renewcommand{\glossaryname}{Glosario}

\pagestyle{fancy}
\fancyhf{}
\fancyhead[LO]{\leftmark}
\fancyhead[RE]{\rightmark}
\fancyhead[RO,LE]{\textbf{\thepage}}
\renewcommand{\chaptermark}[1]{\markboth{\textbf{#1}}{}}
\renewcommand{\sectionmark}[1]{\markright{\textbf{\thesection. #1}}}

\setlength{\headheight}{1.5\headheight}

\newcommand{\HRule}{\rule{\linewidth}{0.5mm}}
%Definimos los tipos teorema, ejemplo y definición podremos usar estos tipos
%simplemente poniendo \begin{teorema} \end{teorema} ...
\newtheorem{teorema}{Teorema}[chapter]
\newtheorem{ejemplo}{Ejemplo}[chapter]
\newtheorem{definicion}{Definición}[chapter]

\definecolor{gray97}{gray}{.97}
\definecolor{gray75}{gray}{.75}
\definecolor{gray45}{gray}{.45}
\definecolor{gray30}{gray}{.94}

\lstset{ frame=Ltb,
     framerule=0.5pt,
     aboveskip=0.5cm,
     framextopmargin=3pt,
     framexbottommargin=3pt,
     framexleftmargin=0.1cm,
     framesep=0pt,
     rulesep=.4pt,
     backgroundcolor=\color{gray97},
     rulesepcolor=\color{black},
     %
     stringstyle=\ttfamily,
     showstringspaces = false,
     basicstyle=\scriptsize\ttfamily,
     commentstyle=\color{gray45},
     keywordstyle=\bfseries,
     %
     numbers=left,
     numbersep=6pt,
     numberstyle=\tiny,
     numberfirstline = false,
     breaklines=true,
   }
 
% minimizar fragmentado de listados
\lstnewenvironment{listing}[1][]
   {\lstset{#1}\pagebreak[0]}{\pagebreak[0]}

\lstdefinestyle{CodigoC}
   {
	basicstyle=\scriptsize,
	frame=single,
	language=C,
	numbers=left
   }
\lstdefinestyle{CodigoC++}
   {
	basicstyle=\small,
	frame=single,
	backgroundcolor=\color{gray30},
	language=C++,
	numbers=left
   }

 
\lstdefinestyle{Consola}
   {basicstyle=\scriptsize\bf\ttfamily,
    backgroundcolor=\color{gray30},
    frame=single,
    numbers=none
   }


\newcommand{\bigrule}{\titlerule[0.5mm]}


%Para conseguir que en las páginas en blanco no ponga cabecerass
\makeatletter
\def\clearpage{%
  \ifvmode
    \ifnum \@dbltopnum =\m@ne
      \ifdim \pagetotal <\topskip
        \hbox{}
      \fi
    \fi
  \fi
  \newpage
  \thispagestyle{empty}
  \write\m@ne{}
  \vbox{}
  \penalty -\@Mi
}
\makeatother

\usepackage{pdfpages}
\begin{document}
\input{portada/portada}
\chapter*{}
%\thispagestyle{empty}
%\cleardoublepage

%\thispagestyle{empty}

\input{portada/portada_2}



\cleardoublepage
\thispagestyle{empty}

\begin{center}
{\large\bfseries \myTitle}\\
\end{center}
\begin{center}
\myName(alumno)\\
\end{center}

%\vspace{0.7cm}
\noindent{\textbf{Palabras clave}: Monotonic, data streams, decision tree, Hoeffding tree, classification, MOA}\\

\vspace{0.7cm}
\noindent{\textbf{Resumen}}\\

En la presente documentación se tratarán diversas temáticas relativas a la ciencia de datos tales como son, los árboles de decisión, los flujos de datos y la clasificación monotónica.\\

 En la primera parte del documento se pondrá al lector en el contexto del problema mediante la explicación de manera detallada de estas técnicas, así como ejemplos de uso, ventajas e inconvenientes, etc.\\
 
  Posteriormente se procederá a la exposición de la propuesta con el fin de hacer entender al lector que consiste en una técnica novedosa en la materia. Esta consiste en una adaptación a un algoritmo existente de árboles de decisión para flujos de datos que además, en este caso, posean restricciones monotónicas para hacer que los modelos aprendidos a partir de los datos sean más fieles a la realidad.\\
  
 Tras la definición de la propuesta se describirán los detalles de los experimentos implementados para la realización de la propuesta, es decir, el \textbf{marco de trabajo}, el cual incluye una descripción de los conjuntos de datos empleados y medidas para la comparativa de algoritmos, \textbf{resultados} de los experimentos y, finalmente, \textbf{análisis} de estos.

Finalmente se finalizará el documento con una serie de conclusiones y posibles trabajos futuros.

\cleardoublepage


\thispagestyle{empty}


\begin{center}
{\large\bfseries \myTitleEng}\\
\end{center}
\begin{center}
\myName (student)\\
\end{center}

%\vspace{0.7cm}
\noindent{\textbf{Keywords}: Monotonic, data streams, decision tree, Hoeffding tree, classification, MOA}\\

\vspace{0.7cm}
\noindent{\textbf{Abstract}}\\

This documentation will cover various topics related to data science such as decision trees, data flows and monotonic classification.

In the first part of the document the reader will be placed in the context of the problem by explaining in detail these techniques, as well as examples of use, advantages and disadvantages, etc. \\

Subsequently, the proposal will be presented in order to make the reader understand that it consists of a new technique in the field. This consists of an adaptation to an existing decision tree algorithm for data flows that also, in this case, have monotonic restrictions to make the models learned from the data more faithful to reality. \\

After defining the proposal, the details of the experiments implemented for the realization of the proposal will be described, that is, the \textbf{framework}, which includes a description of the data sets used and measures for the comparison of algorithms, \textbf{results} of the experiments and, finally, \textbf{analysis} of these.

Finally, the document will be finished with a series of conclusions and possible future work.

\chapter*{}
\thispagestyle{empty}

\noindent\rule[-1ex]{\textwidth}{2pt}\\[4.5ex]

Yo, \textbf{\myName}, alumno de la titulación \myDegree de la \textbf{\myFaculty}, con DNI 20486926K, autorizo la
ubicación de la siguiente copia de mi Trabajo Fin de Máster en la biblioteca del centro para que pueda ser
consultada por las personas que lo deseen.

\vspace{6cm}

\noindent Fdo: \myName

\vspace{2cm}

\begin{flushright}
Granada a 1 de septiembre de 2019.
\end{flushright}


\chapter*{}
\thispagestyle{empty}

\noindent\rule[-1ex]{\textwidth}{2pt}\\[4.5ex]

D. \textbf{\myProf (tutor1)}, Profesor del Departamento de Ciencias de la Computación e I.A. de la Universidad de Granada.

\vspace{0.5cm}

\textbf{Informa:}

\vspace{0.5cm}

Que el presente trabajo, titulado \textit{\textbf{\myTitle}},
ha sido realizado bajo su supervisión por \textbf{\myName}, y autoriza la defensa de dicho trabajo ante el tribunal que corresponda.

\vspace{0.5cm}

Y para que conste, expide y firma el presente informe en Granada a 1 de Septiembre de 2019 .

\vspace{1cm}

\textbf{El director:}

\vspace{5cm}

\noindent \textbf{\myProf}

\chapter*{Agradecimientos}
\thispagestyle{empty}

       \vspace{1cm}

Llegado a este punto, agradezco la paciencia e interés puesto en mi aprendizaje a todos los profesores que han formado parte, durante todo este año, de poner en mis manos la semilla de conocimiento que me servirá para lanzarme al mundo profesional, así como a mi tutor Salvador García López, quien se ha encargado de ayudarme y supervisar este TFG.
Agradezco a toda mi familia y, con mayor énfasis a mis padres y a mi hermano, el interés y el apoyo que me han ofrecido desde el primer momento, aunque no entiendan del todo las ''letras raras'' en la pantalla de mi ordenador cuando trabajo, o que no nos enseñen a ''hackear'' cosas. Por último, agradezco el haber prolongado el contacto en el ámbito académico con los amigos que hice durante el grado, con buena compañía todo ha sido más sencillo, ya sabéis.

\frontmatter
\tableofcontents
\listoffigures
%
\mainmatter
\setlength{\parskip}{5pt}

%Introduccion del Trabajo Fin de Máster
\chapter[Capítulo 1. Introducción y objetivos]{Introducción y objetivos}

El objetivo principal de este documento es, primeramente, dotar al lector de los conocimientos necesarios sobre las distintas técnicas a utilizar para el desarrollo del algoritmo, con el fin de llegar a entender el propósito de este proyecto.\\
En este punto se tratarán los temas de clasificación con restricciones monotónicas, la clasificación sobre flujos de datos, y el background correspondiente a la clasificación mediante el uso de árboles de decisión, así como la combinación de algunas de estas técnicas entre sí, tales como la del uso de árboles de clasificación sobre flujos de datos.

Una vez hayamos dotado al lector de la contextualización del problema, el siguiente objetivo es la descripción inicial de la propuesta, seguido de la explicación del software desarrollado para tal propósito, así como su uso.

Finalmente, aparecerá en este documento una exposición de los experimentos realizados con la propuesta, además de comparaciones de sus resultados con los de los algoritmos que citaremos más adelante, con el propósito de observar si la propuesta cumple el cometido de resultar ser mejor en los aspectos que deseemos.\\
Acompañado de estas comparaciones y para finalizar el documento, presentaremos también una serie de conclusiones y trabajos futuros a desarrollar para continuar con esta línea de trabajo.

\newpage


\chapter[Capítulo 2. Background en problemas]{Background en problemas}

\section{Data streaming classification}

\section{Ordinal and monotonic classification}

\newpage


\chapter[Capítulo 3. Background en algoritmos de árboles de decisión]{Background en algoritmos de árboles de decisión}

\section{Fundamentos de árboles de decisión}

 Los \textbf{árboles de decisión} \cite{ref3} son un tipo de algoritmos de aprendizaje supervisado (tanto para clasificación como para regresión) utilizado en diversos ámbitos como la inteligencia artificial, las finanzas, el marketing, etc. \\ Dado un conjunto de datos se fabrican diagramas de construcciones lógicas en forma de ramificaciones de árboles, muy similares a los sistemas de predicción basados en reglas, que sirven para representar y categorizar una serie de condiciones que ocurren de forma sucesiva, para la resolución de un problema.
 
 Adentrándonos en los aspectos más técnicos de este tipo de modelos de predicción, cabe destacar que las variables de entrada y de salida pueden ser tanto categóricas como continuas y que divide el espacio de los predictores (variables independientes) en regiones distintas y no superpuestas, tal como veremos en la siguiente figura.
 
 \begin{figure}[H]
 	\centering
 	\includegraphics[width=0.7\textwidth]{imagenes/ejemploArboles} 
 	\caption{Ejemplo de árbol de decisión \cite{ref4}}
 \end{figure}
 
 Estas divisiones se realizan creando sobre la población (el conjunto de datos) subconjuntos lo más homogéneos posible entre las muestras que componen un grupo y lo más heterogéneo posible entre los distintos subconjuntos.\\
 Para la efectuación de esta separación, el algoritmo se basa en las variables de entrada más significativas, es decir, las que mejor separan las muestras.
 
 A continuación podemos observar las diferentes partes de un árbol de decisión.
 
  \begin{figure}[H]
 	\centering
 	\includegraphics[width=0.5\textwidth]{imagenes/ejemploArbol} 
 	\caption{Partes de un árbol de decisión \cite{ref5}}
 \end{figure}

\subsection{Terminología:}
\begin{itemize}
	\item \textbf{Nodo raíz}: Es el primero de los nodos del árbol y forma la población completa.
	\item \textbf{Ramificación}: Son las ramas que conectan todos los nodos del árbol por donde pasan las muestras para ser clasificadas.
	\item \textbf{Nodo de decisión}: Son aquellos donde las muestras se evaluan para decidir por qué rama continuar el camino hacia la solución.
	\item \textbf{Nodo terminal/hoja}: Estos son los nodos solución, una vez la muestra llega a este tipo de nodo, el proceso de evaluación de esta ya ha finalizado, por lo que habrá sido clasificada en alguno de los grupos categóricos existentes.
	\item \textbf{Poda}: Consiste en cortar u obviar una rama del árbol en la creación de un árbol, basándonos en cierta propiedad escogida, para evitar el recorrido del árbol completo y ahorrar de esta forma costos computacionales, así como para hacer frente al sobreajuste.
	\item \textbf{Rama/subárbol}: Es el conjunto de nodos y ramas completo que queda estrictamente por debajo de un nodo escogido del árbol total.
	\item \textbf{Nodos padre e hijo}: Dado un nodo del árbol, sus nodos hijo son todos aquellos que quedan conectados directamente a él únicamente en el nivel inferior siguiente. De esta forma, esos nodos hijo, comparten ese mismo padre.
\end{itemize}


\subsection{¿Regresión o clasificación?: similitudes y diferencias.}

\subsubsection{Similitudes}

Ya sabemos, por ejemplo, que un árbol de decisión divide el espacio de los predictores en regiones no solapadas mediante el uso de los predictores más significativos.

Los árboles de decisión actúan bajo la llamada \textbf{separación binaria recursiva}, basada en un método greedy el cual decidirá en cada momento cuál será la mejor separación en el instante actual para encontrar el mejor árbol. El término 'binaria' hace alusión al tipo de división acaecido en cada nodo, es decir, que cada nodo divide en dos el espacio de los predictores. El término 'recursiva' se refiere a que el algoritmo realiza este proceso de forma reiterada hasta llegar a un criterio de parada predefinido.

Este proceso nos conduce a la generación de un árbol completo si no hacemos uso de criterios de parada, lo que nos lleva de forma directa al problema del sobreajuste, obteniendo un modelo de una pésima calidad a la hora de evaluar nuevos datos. Por esto mismo es necesario difinir criterios de parada que realicen podas sobre el árbol para generar modelos lo suficientemente genéricos que eviten ese overfitting.

\subsubsection{Diferencias}

Las diferencias entre ambos modelos son bastante evidentes: para conjuntos de datos donde se utiliza una variable dependiente continua, utilizamos árboles de regresión, mientras que cuando la variable dependiente es categórica, usamos árboles de clasificación.\\
Dado esto, el \textbf{valor de los nodos hoja} no pueden ser calculados de la misma forma para ambas técnicas, por lo que, en \textbf{árboles de regresión}, utilizamos la \textbf{media} del valor de salida de las muestras que caen en dicho nodo hoja, mientras que en \textbf{árboles de clasificación} utilizamos la \textbf{moda} para asignar un valor de salida a nuevas muestras.

\subsection{Ventajas e inconvenientes de los árboles de decisión}

\subsubsection{Ventajas}

\begin{itemize}
	\item Fáciles de comprender a la hora de interpretar los resultados.
	\item El tipo de dato utilizado no es una limitación.
	\item Es un método no paramétrico, es decir, en el que no es necesario hacer suposiciones sobre el espacio de distribución y la estructura del clasificador.
	\item Resulta útil a la hora de detectar la relevancia de los predictores aún habiendo una gran cantidad de estos.
	\item No son influidos por outliers ni valores perdidos (hasta cierto punto), por lo que requieren una menor limpieza de datos en comparación con otros métodos.
\end{itemize}

\subsubsection{Inconvenientes}

\begin{itemize}
	\item Producen sobreajuste, por lo que hay que tener cuidado con ello mediante el uso de restricciones y la aplicación de poda.
	\item A la hora de trabajar con variables continuas, el árbol de decisión pierde información en el momento en el que categoriza dichas variables para la generación del árbol.
	\item No son del todo competentes con los mejores algoritmos de aprendizaje supervisado en cuanto a precisión en la predicción, es decir, no resultan ser tan efectivos ensambladores o SVM por ejemplo.
	\item Son sensibles al ruido en los datos, ya que este puede modificar de forma significativa la estructura del árbol.
\end{itemize}

\subsection{Creación del árbol}

Como ya sabemos, un árbol comienza desde un nodo raíz donde se encuentra clasificada toda la población y, conforme vamos profundizando por las ramas inferiores, vamos obteniendo subconjuntos cada vez más y más homogéneos con respecto a la variable de salida.

Para hacer posible esto necesitamos que nuestro modelo tome, por cada nodo, una decisión de separación de los datos basada en la ganancia de pureza (esa homogeneidad en los subconjuntos) al utilizar uno u otro predictor en cada uno de los nodos de decisión para crear esas particiones del espacio sucesivas.\\
Es decir, para cada nodo, se evalúa mediante unos medidores de pureza, cual es el predictor o característica del conjunto de datos en dicho instante que separa de mejor forma los datos que tenemos en ese momento con respecto a la variable de salida. Se escogerá en cada nivel, el predictor que mayor pureza ofrezca al árbol de decisión.

De esta forma vamos construyendo de manera progresiva ramas y más ramas del árbol haciendo uso de una técnica greedy de selección de característica a evaluar en cada nivel del árbol para la toma de decisión a la hora de generar nuevos nodos hijos.

\subsubsection{¿Cómo medimos esa ganancia de homogeneidad?}

Para los\textbf{ árboles de regresión} sabemos que el objetivo de cada decisión del árbol en la creación de nuevas separaciones es minimizar la función RSS (\textbf{Residual Sum of Squares}), una medida de error usada también en la regresión lineal. Es por ello que en cada nodo se escogerá una forma de particionar los datos mediante el uso de un predictor u otro, atendiendo a cuál minimiza en mayor medida dicha fórmula.

\begin{figure}[H]
	\centering
	\includegraphics[width=0.5\textwidth]{imagenes/rss} 
	\caption{Recursive Binary Splitting in regression trees \cite{ref6}}
\end{figure}

Como el problema que nos concierne en este caso es el de \textbf{árboles de clasificación}, no entraremos en más detalles acerca de esta fórmula, ya que RSS no puede ser utilizado como criterio de separación binaria en este tipo de árboles.

Una aproximación natural a RSS es el '\textbf{ratio de error en la clasificación}', basado simplemente en la fracción de las observaciones de training en dicha región que no pertenecen a la clase más común de esta. Se calcula de la siguiente forma:

\begin{figure}[H]
	\centering
	\includegraphics[width=0.5\textwidth]{imagenes/cer} 
	\caption{Recursive Binary Splitting in classification trees (classification error rate) \cite{ref7}}
\end{figure}

Donde $\widehat{P}$mk es la proporción $\widehat{P}$ de las observaciones de train de la región \textbf{m} que pertenecen a la clase \textbf{k}. Por tanto, el máximo de $\widehat{P}$mk hace alusión a la proporción máxima de elementos de train que siguen la moda en dicha partición.\\
Por desgracia esta medida de error no es lo suficientemente sensible conforme el árbol crece, por lo que es preferible el uso de otras medidas como el índice Gini y la Entropía.

Ya sabemos que buscamos nodos con una distribución de clase lo más homogénea posible. Con el fin de medir esa pureza en cada nodo, podremos utilizar los siguientes métodos ya mencionados:
\begin{itemize}
	\item \textbf{Índice GINI}: nos indica como de pura es una región del espacio. En este caso la pureza la definimos como la proporción de items de la región que pertenecen a una misma clase. Si la región contiene un índice
	alto de pureza, entonces el índice Gini será bajo (muy próximo a 0), siguiendo la siguiente fórmula:
	\begin{figure}[H]
		\centering
		\includegraphics[width=0.5\textwidth]{imagenes/gini} 
		\caption{Recursive Binary Splitting in classification trees (gini) \cite{ref8}}
	\end{figure}
	Donde $\widehat{\pi}$mc nos indica la proporción \textbf{$\widehat{\pi}$} de items pertenecientes a la misma clase \textbf{c} en la región/nodo \textbf{m}.
	\item \textbf{Entropía}: Se encarga también de la medida de homogeneidad de un nodo. En este caso, los resultados obtenidos al aplicar la siguiente fórmula a cada nodo para observar el nivel de entropía, quedan más visibles con respecto a la medida Gini (es decir, se ve de forma más clara el nivel de homogeneidad de un nodo). Una entropía = 0 significa homogeneidad total, una entropia = 1 significa homogeneidad nula.
	\begin{figure}[H]
		\centering
		\includegraphics[width=0.5\textwidth]{imagenes/entropia} 
		\caption{Recursive Binary Splitting in classification trees (entropia) \cite{ref8}}
	\end{figure}
\end{itemize}

\subsubsection{¿Cómo evitamos el overfitting?}

Una vez hemos escogido nuestra estrategia de creación del árbol, necesitamos indicarle al algoritmo cuándo ha de terminar de construirlo el uso de restricciones (prepruning) y su posterior poda (postpruning) para evitar de esta forma un sobreajuste a los datos.\\

\textbf{Prepruning: establecimiento de parámetros}
\begin{itemize}
	\item Definir un número de observaciones mínimo sobre un nodo para que sea considerada una ramificación sobre él.
	\item Definir un número mínimo de observaciones sobre un nodo hoja.
	\item Establecer una profundidad vertical máxima para el árbol.
	\item Limitar el número máximo de nodos hoja.
	\item Parar si la expansión del nodo actual no mejora la medida de pureza utilizada actual.
\end{itemize}

\textbf{Proceso de Postpruning}:
\begin{enumerate}
	\item Crear un árbol muy grande con o sin restricciones de prepruning.
	\item Recorrer el árbol de abajo hacia arriba para ir cortando las hojas que nos dan ganancias negativas.
\end{enumerate}

De esta forma podemos mantener ramas que, sin el proceso de postpruning podrían haber sido recortadas, pero que nos llevan a soluciones mejores que las que se ofrecen si este proceso por culpa del uso de la técnica greedy.


\section{Hoeffding Trees y otros algoritmos de data streaming}

Como ya sabemos, los algoritmos dedicados al data streaming han de seguir los siguientes requisitos:
\begin{itemize}
	\item Procesar una muestra en cada momento y hacerlo tan solo una vez.
	\item Usar una cantidad de memoria limitada.
	\item Trabajar en un tiempo limitado.
	\item Estar listo para la predicción en cualquier momento.
\end{itemize}

Además, nuestro algoritmo ha de estar dotado de técnicas de detección de cambios en la distribución de los datos para evitar la disminución de la precisión en la predicción cuando esto suceda.

\subsection{Hoeffding-trees}

Un árbol de Hoeffding (VFDT) es un algoritmo de inducción de árbol de decisión incremental capaz de aprender de flujos de datos masivos, suponiendo que la distribución que genera ejemplos no cambia con el tiempo. Los árboles Hoeffding explotan el hecho de que una pequeña muestra a menudo puede ser suficiente para elegir un atributo de división óptimo. Esta idea está respaldada matemáticamente por el Hoeffding bound, que cuantifica el número de observaciones (en nuestro caso, ejemplos) necesarias para estimar algunas estadísticas dentro de una precisión prescrita (en nuestro caso, la bondad de un atributo). \cite{ref9}

Algunas de las técnicas de clasificación para data streaming tienen los siguientes \textbf{problemas}:
\begin{itemize}
	\item Son altamente sensibles a la demanda de ejemplos.
	\item Carecen de alta eficiencia, siendo en algunos casos más lentos que un algoritmo batch.
\end{itemize}

Ante estos problemas se plantea\textbf{ Hoeffding-tree} ya que:
\begin{itemize}
	\item El aprendizaje de un Hoeffding-tree toma un tiempo constante en cada nuevo ejemplo, lo que lo hace adecuado para el aprendizaje de flujos de datos.
	\item Los árboles resultantes son similares a los creados con un batch learner convencional.
\end{itemize}

\subsubsection{Hoeffding bound.}

Con el fin de cumplir con los requisitos establecidos al principio de este apartado para el tratamiento de fllujos de datos, los autores (Hulten y Domingos) proponen la cota Hoeffding para ser capaces de decidir la cantidad de instancias necesarias a evaluar para alcanzar un cierto nivel de confianza a partir del cual sabemos que no es necesario evaluar más ejemplos para seleccionar un atributo mediante el cual realizar la partición del árbol en el nodo actual.\\
Es decir, una vez alcanzada la cota de Hoeffding, el atributo que seleccionemos para el particionamiento del espacio de predictores, sera el mismo que seleccionariamos si analizásemos una infinidad de ejemplos con el clasificador (evidentemente, con cierto nivel de confianza).

\section{Árboles de decisión monotónicos}


\newpage



\chapter[Capítulo 4. Propuesta]{Propuesta}

\section{Introducción}

Una vez conocemos el funcionamiento, características, algunas formas de implementación y los problemas a los que se enfrentan cada una de las técnicas descritas hasta ahora (es decir, de los árboles de decisión, el data streaming y la aplicación de restricciones monotónicas), procedemos a exponer la propuesta hacia la cual se encauza el propósito de este documento.\\

Hemos comprobado que existen métodos eficientes de adquisición de conocimiento y predicción sobre flujos de datos utilizando árboles de decisión, tales como Option trees y Hoeffding Anytime Tree (HATT), ambos basados en Hoeffding Tree, o incluso su versión mejorada para la detección del cambio de concepto en el tiempo (CVFDT).\\

Así mismo, hemos observado también que existen ejemplos de conjuntos de datos que poseen características dignas de ser tenidas en cuenta a la hora de crear un modelo de aprendizaje de conocimiento, ya que aportan una base de información subyacente a los datos que ofrece al algoritmo la capacidad de crear una estructura de datos más eficiente para la resolución de cualquier problema que se enfrente a datos de este estilo. Evidentemente hablamos de los conjuntos de datos que aguardan en sí una relación de monotonía entre los valores de los atributos de sus instancias y las clases asignadas a estas.\\

Tal como se muestra en \cite{ref14}, existen multitud de problemas en la vida diaria que requieren ser tratados como problemas de clasificación monotónica con el fin de ser correctamente resueltos. Este es el caso de aquella universidad que no quiera admitir a un solicitante con ciertas notas y rechazar a otro con notas iguales o más altas por no haber tenido en cuenta la monotonicidad del asunto, o el caso de una compañía de seguros de vida que no desea que sus decisiones dependan de un árbol de decisión que no tenga en cuenta que un solicitante anciano y poco saludable ha de cotizar una tasa de prima más alta que un solicitante joven y saludable. Otros problemas pueden ser los de credit scoring, elección de consumidor, selección de escuela y transporte, etc.\\

En el siguiente ejemplo veremos las consecuencias que puede tener el hecho de crear un árbol de clasificación no-monotónico en un problema de credit scoring simple. Los atributos utilizados en cada árbol son los activos + los ingresos y los activos a solas respectivamente. 


\begin{figure}[H]
	\centering
	\includegraphics[width=1\textwidth]{imagenes/ejAr} 
	\caption{Árboles no monotónicos\cite{ref14}}
\end{figure}

Como vemos, ambos árboles, al no poseer restricciones de monotonicidad, carecen también de sentido, ya que a un cliente con pocos activos se le autoriza una linea de crédito de cinco mil dólares, mientras que a un cliente con más activos no se le ofrece crédito alguno, por ejemplo.

A la derecha podemos observar el árbol resultante aplicado a un conjunto de datos que \textbf{sí} guarda relación de monotonía entre los valores de los atributos y las clases asignadas y que, tal como vemos, no asegura que el árbol creado a partir de él sea monotónico.\\
Los datos utilizados para ello son los siguientes:

\begin{figure}[H]
	\centering
	\includegraphics[width=0.9\textwidth]{imagenes/tablaEj} 
	\caption{Conjunto de datos monotónico\cite{ref14}}
\end{figure}

\section{Propuesta y resultados esperados}

Visto todo esto, parece lógico crear una adaptación de los algoritmos de árboles de decisión para data streaming que además posean información sobre restricciones de monotonía para conseguir resultados superiores en cuanto al análisis del dominio del problema al que se enfrentan los algoritmos que trabajan con este tipo de conjuntos de datos.\\

La propuesta no pretende conseguir resultados mejores en cuanto al accuracy obtenido en la predicción de resultados si no que, pretende que las estructuras creadas por los árboles de decisión para flujos de datos sean más fieles a la realidad que subyace bajo data sets con cierto nivel de monotonía en su relación atributos-clase.\\

Dicho esto, podría darse el caso en el que el algoritmo básico comparativo sin restricciones de monotonía alcance un accuracy superior al que puedan alcanzar las adaptaciones realizadas las cuales si posean información sobre dichas relaciones monotónicas.\\

\section{Algoritmos a comparar}

Como \textbf{algoritmo de comparación} utilizaremos aquél del cual la literatura ha hecho más uso para la realización de estudios comparativos y mejoras de otros aspectos, es decir: \textbf{Hoeffding Tree}, la base de la que parten gran cantidad de algoritmos de árboles de decisión para el tratamiento de data streaming.\\

Los \textbf{algoritmos que pretenden hacerle frente} en cuanto a los aspectos ya mencionados serán:
\begin{itemize}
	\item Adaptación de Hoeffding Tree con restricciones monotónicas mediante el uso del \textbf{método basado en matriz}.
	\item Adaptación de Hoeffding Tree con restricciones monotónicas mediante el uso del \textbf{método de evaluación de monotonicidad de ramas}.
	\item Adaptación de Hoeffding Tree con restricciones monotónicas mediante el uso de\textbf{ ambos métodos al mismo tiempo}.
\end{itemize}











\newpage



\chapter[Capítulo 5. Software desarrollado y uso]{Software desarrollado y uso}

El software para la realización de los experimentos ha sido creado mediante el IDE de programación Netbeans con Java como lenguaje de programación.

Los datos empleados son estáticos, por lo que ha sido necesario adaptar el código fuente a un entorno de flujo de datos para poder realizar los experimentos.

Como ya se ha comentado, el algoritmo de partida de los experimentos es el Hoeffding Tree, el cual se basa en una adaptación de árboles de decisión para flujos de datos. Este código ha sido tomado de la API de MOA.

MOA (Massive Online Analisys) \cite{ref31} es un entorno de trabajo open-source relacionado con el proyecto WEKA (Waikato Environment for Knowledge Analysis) para el tratamiento o minería de flujos de datos de evolución masiva conteniendo una gran colección de algoritmos de machine
learning como son: clasificación, regresión, clustering, detección de outliers, detección de concept drift y sistemas de recomendación, así como herramientas de evaluación. 

Los \textbf{algoritmos implementados} son los descritos en la propuesta:
\begin{itemize}
	\item Algoritmo 1: Árboles de clasificación monotónica sobre flujos de datos. Método de matriz de colisiones.
	\item Algoritmo 2: Árboles de clasificación monotónica sobre flujos de datos. Método de evaluación a la hora de expandir el árbol.
\end{itemize}

Para la obtención de resultados y la posible comparativa de estos entre los distintos modelos, para ambos algoritmos ha sido necesario modificar algunos de los parámetros de Hoeffding Tree:
\begin{itemize}
	\item \textbf{Desempate(tie-breaking)}: este parámetro ayuda, tal como su propio nombre indica, a evitar estancamientos en las expansiones de los árboles acaecidos por la similitud entre la ganancia ofrecida por dos atributos a la hora de decidirse por alguno para expandir. Si son muy similares, la diferencia entre ellos sera pequeña, por tanto será imposible que supere la cota Hoeffding, lo cual es necesario para poder expandir. Gracias a su modificación, he podido obtener árboles más extensos, ya que sin tocarlo, el pequeño volumen de datos manejado no lo permitía.
	\item \textbf{Grace period option}: con él indicamos la cantidad de instancias necesarias para la generación de nodos nuevos en el árbol.
\end{itemize}

Dichos han sido creados adaptando el código y resultados de Hoeffding Tree a las necesidades de cada uno y, los resultados de estos han sido sometidos a\textbf{ validación cruzada de 10 iteraciones} para obtener resultados más fieles al comportamiento de cada algoritmo. Para obtener los resultados de cada iteración de la validación cruzada he creado mis propias\textbf{ funciones de evaluación} de la medida de \textbf{accuracy (MAE en este caso)} y del \textbf{índice de no-monotonicidad (NMI).}

\textbf{Para la obtención de} la medida \textbf{MAE}, como ya hemos visto, simplemente divido la suma de errores de cada iteración entre la cantidad de elementos en el conjuntos de datos. 

\textbf{Para obtener el NMI} hago uso de los datos predichos por el algoritmo de la siguiente forma:
\begin{enumerate}
	\item Detecto las colisiones entre las distintas instancias (es decir, compruebo con cuantas instancias no aguarda monotonía respecto a atributos y clase cada una de las instancias del conjunto de datos).
	\item Realizo la suma de dichas colisiones.
	\item Dividio entre la cantidad máxima de colisiones que podrían ocurrir en dicho conjunto de datos.
\end{enumerate}

El \textbf{proceso de aprendizaje del modelo y validación del mismo} para cada iteración de la validación cruzada es el siguiente:
\begin{enumerate}
	\item Entrenamiento del modelo con el algoritmo escogido de los tres implementados haciendo uso de los datos de entrenamiento a modo de flujo de datos.
	\item Generación de predicciones con los datos de test.
	\item Validación de las predicciones en cuanto a accuracy (MAE) y monotonicidad (NMI).
\end{enumerate}

Estos resultados obtenidos son los que se expondrán más adelante en la comparativa de algoritmos.

\newpage



\chapter[Capítulo 6. Experimentos]{Experimentos}

\section{Framework}

\section{Resultados}

\section{Análisis}

\newpage












\chapter[Capítulo 7. Conclusiones y trabajo futuro]{Conclusiones y trabajo futuro}

Finalmente, una vez explicado el desarrollo completo del proyecto, es el momento de mirar hacia atrás para cerciorarme de las asignaturas del grado cursadas sin las que no hubiera sido capaz de desenvolverme de manera tan efectiva para la realización del proyecto, así como darme cuenta de los conocimientos que he adquirido durante esta etapa de mi carrera y, por último, realizar una valoración de todo ello.


\newpage












%
%
%%\nocite{*}
\bibliography{citas} %archivo citas.bib que contiene las entradas 
\bibliographystyle{plain} % hay varias formas de citar

%\input{capitulos/Anexo}

%
%\appendix
%\input{apendices/manual_usuario/manual_usuario}
%%\input{apendices/paper/paper}
%\input{glosario/entradas_glosario}
% \addcontentsline{toc}{chapter}{Glosario}
% \printglossary
\chapter*{}
\thispagestyle{empty}

\end{document}
