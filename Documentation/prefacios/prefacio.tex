\chapter*{}
%\thispagestyle{empty}
%\cleardoublepage

%\thispagestyle{empty}

\input{portada/portada_2}



\cleardoublepage
\thispagestyle{empty}

\begin{center}
{\large\bfseries \myTitle}\\
\end{center}
\begin{center}
\myName(alumno)\\
\end{center}

%\vspace{0.7cm}
\noindent{\textbf{Palabras clave}: Monotonic, data streams, decision tree, Hoeffding tree, classification, MOA}\\

\vspace{0.7cm}
\noindent{\textbf{Resumen}}\\

En la presente documentación se tratarán diversas temáticas relativas a la ciencia de datos tales como son, los árboles de decisión, los flujos de datos y la clasificación monotónica.\\

 En la primera parte del documento se pondrá al lector en el contexto del problema mediante la explicación de manera detallada de estas técnicas, así como ejemplos de uso, ventajas e inconvenientes, etc.\\
 
  Posteriormente se procederá a la exposición de la propuesta con el fin de hacer entender al lector que consiste en una técnica novedosa en la materia. Esta consiste en una adaptación a un algoritmo existente de árboles de decisión para flujos de datos que además, en este caso, posean restricciones monotónicas para hacer que los modelos aprendidos a partir de los datos sean más fieles a la realidad.\\
  
 Tras la definición de la propuesta se describirán los detalles de los experimentos implementados para la realización de la propuesta, es decir, el \textbf{marco de trabajo}, el cual incluye una descripción de los conjuntos de datos empleados y medidas para la comparativa de algoritmos, \textbf{resultados} de los experimentos y, finalmente, \textbf{análisis} de estos.

Finalmente se finalizará el documento con una serie de conclusiones y posibles trabajos futuros.

\cleardoublepage


\thispagestyle{empty}


\begin{center}
{\large\bfseries \myTitleEng}\\
\end{center}
\begin{center}
\myName (student)\\
\end{center}

%\vspace{0.7cm}
\noindent{\textbf{Keywords}: Monotonic, data streams, decision tree, Hoeffding tree, classification, MOA}\\

\vspace{0.7cm}
\noindent{\textbf{Abstract}}\\

This documentation will cover various topics related to data science such as decision trees, data flows and monotonic classification.

In the first part of the document the reader will be placed in the context of the problem by explaining in detail these techniques, as well as examples of use, advantages and disadvantages, etc. \\

Subsequently, the proposal will be presented in order to make the reader understand that it consists of a new technique in the field. This consists of an adaptation to an existing decision tree algorithm for data flows that also, in this case, have monotonic restrictions to make the models learned from the data more faithful to reality. \\

After defining the proposal, the details of the experiments implemented for the realization of the proposal will be described, that is, the \textbf{framework}, which includes a description of the data sets used and measures for the comparison of algorithms, \textbf{results} of the experiments and, finally, \textbf{analysis} of these.

Finally, the document will be finished with a series of conclusions and possible future work.

\chapter*{}
\thispagestyle{empty}

\noindent\rule[-1ex]{\textwidth}{2pt}\\[4.5ex]

Yo, \textbf{\myName}, alumno de la titulación \myDegree de la \textbf{\myFaculty}, con DNI 20486926K, autorizo la
ubicación de la siguiente copia de mi Trabajo Fin de Máster en la biblioteca del centro para que pueda ser
consultada por las personas que lo deseen.

\vspace{6cm}

\noindent Fdo: \myName

\vspace{2cm}

\begin{flushright}
Granada a 1 de septiembre de 2019.
\end{flushright}


\chapter*{}
\thispagestyle{empty}

\noindent\rule[-1ex]{\textwidth}{2pt}\\[4.5ex]

D. \textbf{\myProf (tutor1)}, Profesor del Departamento de Ciencias de la Computación e I.A. de la Universidad de Granada.

\vspace{0.5cm}

\textbf{Informa:}

\vspace{0.5cm}

Que el presente trabajo, titulado \textit{\textbf{\myTitle}},
ha sido realizado bajo su supervisión por \textbf{\myName}, y autoriza la defensa de dicho trabajo ante el tribunal que corresponda.

\vspace{0.5cm}

Y para que conste, expide y firma el presente informe en Granada a 1 de Septiembre de 2019 .

\vspace{1cm}

\textbf{El director:}

\vspace{5cm}

\noindent \textbf{\myProf}

\chapter*{Agradecimientos}
\thispagestyle{empty}

       \vspace{1cm}

Llegado a este punto, agradezco la paciencia e interés puesto en mi aprendizaje a todos los profesores que han formado parte, durante todo este año, de poner en mis manos la semilla de conocimiento que me servirá para lanzarme al mundo profesional, así como a mi tutor Salvador García López, quien se ha encargado de ayudarme y supervisar este TFG.
Agradezco a toda mi familia y, con mayor énfasis a mis padres y a mi hermano, el interés y el apoyo que me han ofrecido desde el primer momento, aunque no entiendan del todo las ''letras raras'' en la pantalla de mi ordenador cuando trabajo, o que no nos enseñen a ''hackear'' cosas. Por último, agradezco el haber prolongado el contacto en el ámbito académico con los amigos que hice durante el grado, con buena compañía todo ha sido más sencillo, ya sabéis.
