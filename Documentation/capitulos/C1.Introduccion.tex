\chapter[Capítulo 1. Introducción y objetivos]{Introducción y objetivos}

\section{Introducción}

¿Qué es lo primero que se nos viene a la cabeza cuando escuchamos la palabra \textbf{informática}?¿Un ordenador?¿Datos de infinidad de temas y tipos almacenados?¿Programas para facilitar el uso de esos datos?¿Internet?\\

Si mezclamos estas ideas que nos surgen de forma repentina al pensar en dicha palabra, podemos extraer una definición apropiada con facilidad. Una de tantas definiciones para ello, es que la informática se trata de la ciencia que se encarga del almacenamiento, procesamiento y transferencia de información en formato digital a través de un sistema físico con capacidades computacionales (hardware). Mediante dicho \textbf{hardware}, podemos crear programas (\textbf{software}) que faciliten la automatización del proceso de almacenamiento, procesamiento y transferencia de información citados.\\
¿Quién no ha utilizado un ordenador en alguna ocasión para jugar a su juego favorito, imprimir un documento, buscar información por Internet o para ver un vídeo en Youtube de su cantante favorito?\\
Todas estas acciones realizadas con un ordenador (por ejemplo) resultan ser obras de personas con conocimientos específicos en informática necesarios para la creación de estos complejos sistemas dedicados al tratamiento de información de una u otra forma distinta.\\

Si nos adentramos un pequeño paso más hacia aspectos técnicos de la informática, podremos hablar del concepto de \textbf{programación} que, a rasgos muy generales, es el proceso mediante el cual una persona (programador) crea una especie de receta con instrucciones (código fuente) que el ordenador deberá seguir paso a paso para la realización de la tarea deseada por el programador y, es de esta manera la forma de la que se crea un programa.\\

Una vez tenemos la forma de crear procesos automáticos para facilitar a las personas el tratamiento de información, surgen a lo largo de la historia de la informática infinidad de aplicaciones con la que explotar estas capacidades computacionales para sacar provecho económico (empresas privadas), moral (sistemas de vigilancia), académico (clases online) y un largo etcétera de ámbitos de aplicación distintos.\\
Es aquí donde introduzco el concepto clave para la descripción del tema a tratar en este proyecto: \textbf{la Ciencia de Datos}, que en términos muy generales, es un campo interdisciplinar que, a través del uso de ciencias como las matemáticas, estadística e informática, se dedica al análisis y procesamiento de datos para la extracción de información útil de conjuntos de datos(Data Mining), toma de decisiones con respecto a dichos análisis(Estadística inferencial), y automatización de dichos procesos de toma de decisiones(Machine Learning).\\

Como podemos imaginar, poseer conocimientos sobre esta ciencia, supone grandes beneficios para, por ejemplo, empresas que tienen en sus dominios ingentes cantidades de datos sobre sus productos o clientes que los consumen, ya que puede aportarle información para la mejora de la calidad de los productos, mejora de servicios para satisfacer a los clientes, conocimientos sobre el tipo de cliente al que se orienta el producto para crear estrategias de marketing...en conclusión, infinidad de aplicaciones que permiten analizar datos, extraer conclusiones y crear programas automáticos de toma de decisión que un ser humano sería incapaz de realizar en tan poco tiempo y de una forma tan efectiva. Ejemplos de uso de esta ciencia son los de Uber, con su sistema de optimización de ruta en las ciudades, o Amazon, con su sistema de recomendación basado en la predicción de necesidades de sus clientes en función de sus historiales de compra.\\

Para la realización de estos procesos de análisis y procesamiento de datos es indispensable la utilización de algoritmos (esas recetas de las que hablábamos antes) y técnicas complejas de estructuración y análisis de los datos que poseemos en bruto recolectados de alguna manera, que faciliten dichas tareas tanto en tiempo de ejecución como en precisión en la toma de decisiones. Una de estas técnicas de aprendizaje a partir de datos son los llamados \textbf{árboles de decisión}, llamados así porque su estructura (similar a la de un árbol al revés con raíz, ramas y hojas) permite a una computadora ''\textbf{aprender}'' conocimientos sobre un conjunto de datos organizado y \textbf{tomar decisión} en base a ese conocimiento adquirido mediante los ejemplos analizados.\\

Los datos tratados por estas astutas técnicas de análisis no siempre permanecen almacenados y ordenados previo análisis, si no que pueden provenir de fuentes que generan ingentes cantidades de datos de forma periódica y sin pausa, los cuales han de ser tratados de forma rápida y constante para poder obtener una buena representación de su comportamiento y poder generar una toma de decisión correcta. Estos son los llamados \textbf{flujos de datos o data streaming}.\\

Es bien sabido también que, toda ayuda que el ser humano pueda poner de su parte para la mejora de los algoritmos de análisis y toma de decisiones, será bien recibida por cualquiera de las técnicas utilizadas en la Ciencia de Datos. Es esta la idea de hacer uso de las llamadas \textbf{restricciones monotónicas} de las cuales hablaremos durante todo el documento. Consiste básicamente en aportar información subyacente a un tipo problema específico para ayudar al algoritmo a alcanzar mejores resultados.

\section{Motivación}

Tras el estudio a lo largo del máster de todo lo detallado en el apartado anterior y tras ver ejemplos y ejemplos de uso de los citados árboles de decisión, técnicas de data streaming y técnicas de uso de restricciones monotónicas, no he sido capaz de encontrar artículos ni ejemplos donde se haga uso de estos tres elementos de forma conjunta con el fin de obtener mejores resultados sobre flujos de datos con capacidades de obedecer restricciones monotónicas mediante el uso de árboles de decisión.

 En particular, con respecto a las técnicas de flujos de datos, se puede observar en la literatura la inexistencia del tratamiento de estos usando árboles de decisión con restricciones de monotonía.\\

Como bien he dicho anteriormente, toda ayuda que el ser humano pueda aportar a los algoritmos que se encargan de la resolución de cualquier problema, será bien recibida, por tanto, pienso que incluir conocimiento mediante restricciones de monotonía sobre los datos a un algoritmo que utiliza árboles de decisión para resolver problemas de data streaming hará que se obtengan mejores resultados.\\\\

De esta manera, la motivación principal es aportar a la comunidad científica un novedoso tratamiento de algoritmos ya existentes para mejorar la calidad de estos en problemas específicos relativos a la Ciencia de Datos.

\section{Objetivos y estructuración de la memoria}


El objetivo inicial de este documento es dotar al lector de los conocimientos y el contexto necesarios sobre las distintas técnicas a utilizar para que logre entender la finalidad de la propuesta (Capítulos 2 y 3). En esta primera parte del documento se explicarán en profundidad los elementos ya mencionados a formar parte de objeto de estudio:

\begin{itemize}
	\item Clasificación sobre flujos de datos.
	\item Clasificación con restricciones monotónicas.
	\item Árboles de decisión y su uso con data streams y restricciones monotónicas (por separado, evidentemente).
\end{itemize}


Una vez hayamos dotado al lector de la contextualización del problema, el siguiente objetivo (Capítulo 4) es la descripción detallada de la propuesta con el fin de hacer entender las ideas necesarias para la creación del nuevo algoritmo, los pasos a seguir para lograrlo y los resultados esperados con ello, seguido de la explicación del software utilizado para tal propósito (Capítulo 5), así como su uso.\\

Finalmente, aparecerá en este documento una exposición de los experimentos realizados con la propuesta (Capítulo 6), así como la descripción del marco de trabajo en el que lo situaremos, además de comparaciones de sus resultados con los de los algoritmos que citaremos más adelante, con el propósito de observar si la propuesta cumple el cometido de resultar ser mejor en los aspectos que deseemos.\\

Acompañado de estas comparaciones y para finalizar el documento, presentaremos también una serie de conclusiones y trabajos futuros a desarrollar para continuar con esta línea de trabajo (Capítulo 7).

\newpage

