\chapter[Capítulo 7. Conclusiones y trabajo futuro]{Conclusiones y trabajo futuro}

Como conclusión del estudio completo realizado sobre introducción de restricciones monotónicas a los algoritmos de clasificación de flujos de datos con árboles de decisión para el tratamiento de conjuntos de datos con este tipo de relaciones entre los atributos y la variable de salida, hemos podido comprobar que, en efecto, resulta útil para obtener modelos más fieles al problema que se trata con dicho tipo de algoritmos, pero tampoco han parecido tener un éxito demasiado significante si hablamos de números, al menos con los conjuntos de datos empleados.

Es por aquí por donde merece la pena, incluso es necesario, continuar este objeto de estudio, es decir, tratar de ver los resultados de estos algoritmos con flujos de datos reales, donde el volumen masivo nos permitirá comprobar su real efectividad y nos dirá si es necesaria la aplicación de esta novedosa técnicas sobre problemas reales. Es muy posible que al utilizar flujos de datos reales los resultados sean más significativos y consigamos una brecha mayor en cuanto a índice de monotonicidad entre el algoritmo base (sin uso de restricciones) y los demás creados con este fin.

\newpage









