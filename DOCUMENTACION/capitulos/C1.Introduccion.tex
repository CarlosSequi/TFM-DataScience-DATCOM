\chapter[Capítulo 1. Introducción y objetivos]{Introducción y objetivos}

\section{Introducción}

¿Qué es lo primero que se nos viene a la cabeza cuando escuchamos la palabra \textbf{informática}?¿Un ordenador?¿Datos de infinidad de temas y tipos almacenados?¿Programas para facilitar el uso de esos datos?¿Internet?\\

Si mezclamos estas ideas que nos surgen de forma repentina al pensar en dicha palabra, podemos extraer una definición apropiada con facilidad. Una de tantas definiciones para ello, es que la informática se trata de la ciencia que se encarga del almacenamiento, procesamiento y transferencia de información en formato digital a través de un sistema físico con capacidades computacionales (hardware). Mediante dicho \textbf{hardware}, podemos crear programas (\textbf{software}) que faciliten la automatización del proceso de almacenamiento, procesamiento y transferencia de información citados.\\
¿Quién no ha utilizado un ordenador en alguna ocasión para jugar a su juego favorito, imprimir un documento, buscar información por Internet o para ver un vídeo en Youtube de su cantante favorito?\\
Todas estas acciones realizadas con un ordenador (por ejemplo) resultan ser obras de personas con conocimientos específicos en informática necesarios para la creación de estos complejos sistemas dedicados al tratamiento de información de una u otra forma distinta.\\

Si nos adentramos un pequeño paso más hacia aspectos técnicos de la informática, podremos hablar del concepto de \textbf{programación} que, a rasgos muy generales, es el proceso mediante el cual una persona (programador) crea una especie de receta con instrucciones (código fuente) que el ordenador deberá seguir paso a paso para la realización de la tarea deseada por el programador y, es de esta manera la forma de la que se crea un programa.\\

Una vez tenemos la forma de crear procesos automáticos para facilitar a las personas el tratamiento de información, surgen a lo largo de la historia de la informática infinidad de aplicaciones con la que explotar estas capacidades computacionales para sacar provecho económico (empresas privadas), moral (sistemas de vigilancia), académico (clases online) y un largo etcétera de ámbitos de aplicación distintos.\\
Es aquí donde introduzco el concepto clave para la descripción del tema a tratar en este proyecto: \textbf{la Ciencia de Datos}, que en términos muy generales, es un campo interdisciplinar que, a través del uso de ciencias como las matemáticas, estadística e informática, se dedica al análisis y procesamiento de datos para la extracción de información útil de conjuntos de datos(Data Mining), toma de decisiones con respecto a dichos análisis(Estadística inferencial), y automatización de dichos procesos de toma de decisiones(Machine Learning).

Como podemos imaginar, poseer conocimientos sobre esta ciencia, supone grandes beneficios para, por ejemplo, empresas que tienen en sus dominios ingentes cantidades de datos sobre sus productos o clientes que los consumen, ya que puede aportarle información para la mejora de la calidad de los productos, mejora de servicios para satisfacer a los clientes, conocimientos sobre el tipo de cliente al que se orienta el producto para crear estrategias de marketing...en conclusión, infinidad de aplicaciones que permiten analizar datos, extraer conclusiones y crear programas automáticos de toma de decisión que un ser humano sería incapaz de realizar en tan poco tiempo y de una forma tan efectiva. Ejemplos de uso de esta ciencia son los de Uber, con su sistema de optimización de ruta en las ciudades, o Amazon, con su sistema de recomendación basado en la predicción de necesidades de sus clientes en función de sus historiales de compra.

\section{Objetivos}

\section{Motivación}

\section{Estructuración de la memoria}


























El objetivo principal de este documento es, primeramente, dotar al lector de los conocimientos necesarios sobre las distintas técnicas a utilizar para el desarrollo del algoritmo, con el fin de llegar a entender el propósito de este proyecto.\\
En este punto se tratarán los temas de clasificación con restricciones monotónicas, la clasificación sobre flujos de datos, y el background correspondiente a la clasificación mediante el uso de árboles de decisión, así como la combinación de algunas de estas técnicas entre sí, tales como la del uso de árboles de clasificación sobre flujos de datos.

Una vez hayamos dotado al lector de la contextualización del problema, el siguiente objetivo es la descripción inicial de la propuesta, seguido de la explicación del software desarrollado para tal propósito, así como su uso.

Finalmente, aparecerá en este documento una exposición de los experimentos realizados con la propuesta, además de comparaciones de sus resultados con los de los algoritmos que citaremos más adelante, con el propósito de observar si la propuesta cumple el cometido de resultar ser mejor en los aspectos que deseemos.\\
Acompañado de estas comparaciones y para finalizar el documento, presentaremos también una serie de conclusiones y trabajos futuros a desarrollar para continuar con esta línea de trabajo.

\newpage

